\chapter{An\'alisis temporal y costes de desarrollo}\label{anatemporal}

\section{An\'alisis temporal}
	En los apartados anteriores hemos descrito fielmente todo el proceso que ha dado como resultado la creaci\'on 
	de la clase \LaTeX{} \texttt{pclass}. Una vez cubierta la descripci\'on del proceso, pasaremos a tratar en detalle
	la planificaci\'on seguida para la realizaci\'on de este proyecto, as\'i como un desglose de las labores m\'as 
	importantes. Dicha tarea la desarrollaremos a lo largo del cap\'itulo que nos ocupa en este instante.
	
	Para comenzar con el an\'alisis temporal, hemos de distinguir todas las etapas que nos han ocupado durante la 
	elaboraci\'on de este proyecto. Un desglose de estas etapas ser\'ia:
		
	\begin{enumerate}
			\item \textbf{Aprendizaje y documentaci\'on sobre \LaTeX{}}.\\ 
						Al comienzo del desarrollo la soluci\'on al problema propuesto, formateo de memorias para proyectos de 
						fin de carrera, y su implementaci\'on usando lenguaje \LaTeX{} era completamente desconocida 
						para ambos autores del proyecto. Para hacer esta afirmaci\'on nos basamos en que ninguno de los autores 
						gozaba de conocimiento previos acerca de \LaTeX{}, por lo que la construcci\'on de esta clase constitu\'ia
						nuestro primer contacto con dicho lenguaje de programaci\'on. Debido a lo anterior resultaba indispensable
						un periodo inicial de aprendizaje y asimilaci\'on de conceptos b\'asicos sobre \LaTeX{}, as\'i como 
						una b\'usqueda extensa de documentaci\'on a la cual poder acudir en cualquier fase del proyecto.
						
			\item \textbf{Creaci\'on de la clase \texttt{pclass.cls}}\\
						Una vez hemos adquirido los conocimientos necesarios durante la fase anterior, en esta etapa procedemos
						a la creaci\'on en s\'i misma de la clase \texttt{pclass.cls}.
			
			\item \textbf{Fase de pruebas y testeo de \textbf{pclass}}\\
						Las constantes ideas que surg\'ian para dar soluci\'on a los problemas que nos encontrabamos, dieron como 
						resultado que la clase estuviese en todo momento sujeta a gran cantidad de cambios. Por esta raz\'on resultaba
						imprescindible un periodo de pruebas y testeo, para as\'i validar los nuevos elementos introducidos y
						su correcto funcionamiento con los elementos ya existentes.
						
						Adem\'as una vez obtenido el contenido definitivo de \texttt{pclass.cls}, era necesario la creaci\'on de 
						la plantilla para una memoria de proyecto, a la cual aplicar\'iamos el formato definido en el \texttt{.cls}. 
						La finalidad de esta plantilla no es m\'as que facilitar al usuario la redacci\'on de su memoria, olvid\'andose 
						del formato de la misma. Es decir, el usuario har\'a uso de la plantilla y la rellenar\'a con los datos de 
						su memoria, para finalmente obtener como resultado una memoria acorde al formato establecido para la 
						presentaci\'on de la misma.
						
			\item \textbf{Elaboraci\'on de los cap\'itulos de esta memoria}\\
						Durante esta fase hemos realizado la redacci\'on de los cap\'itulos que conforman la memoria de este proyecto. 
										  
			\item \textbf{Creaci\'on del estilo bibliogr\'afico \texttt{pfcbibstyle}}\\
						Durante esta etapa hemos creado, haciendo uso de \texttt{Custom-bib}, un nuevo y propio estilo bibliogr\'afico, 
						el cual aplicaremos a las referencias de esta memoria. Dicho estilo recibe el nombre de pfcbibstyle. Hubiera
						resultado mas sencillo emplear algunos de los estilos existentes m\'as comunes. Sin embargo esta creaci\'on
						nos permitir\'a solucionar algunos problemas derivados del ingl\'es como la ordenaci\'on err\'onea por orden
						alfab\'etico de la letra e\~ne. As\'i mismo nos permitir\'a agrupar las referencias por orden alfab\'etico
						en lugar de por orden de aparici\'on como suele ser m\'as habitual, adem\'as en las citas no aparecer\'a un
						n\'umero sino las inciales del autor o autores y el a\~no de publicaci\'on.  
			
\end{enumerate}


	La planificaci\'on temporal consistir\'a en una estimaci\'on del tiempo dedicado a cada una de las fases
	de desarrollo del proyecto. Para ello asignaremos a cada una de esas fases las dos estimaciones siguientes:
	
\begin{enumerate}
	\item \textbf{Estimaci\'on Inicial} la cual es empleada en los inicios del desarrollo. Suele ser poco exacta,
				pero se usan como primera aproximaci\'on para la viabilidad del proyecto.

	\item \textbf{Estimaci\'on Final} la cual expresa la duraci\'on y el esfuerzo real empleado.
\end{enumerate}
 
	Para realizar una evaluaci\'on de la exactitud de la estimaci\'n se realiza una comparaci\'on de los valores reales 
	con los valores estimados, teniendo en cuenta el error relativo medio. Tomando $RE =\frac{A - E}{A}$, donde A 
	representa el valor real y E el valor estimado previamente, calculamos el error relativo medio mediante la 
	expresi\'on: \\
	
	\begin{displaymath}
     \left(\frac{1}{n}\right) \sum_{i=1}^{n} RE   
	\end{displaymath} \\
	
	
	Las estimaciones de cada fase se han realizado en d\'ias, considerando una dedicaci\'on de dos personas durante una
	media de 4 horas al d\'ia. Dicho esto obtendr\'iamos las estimaciones mostradas en la Tabla~\ref{tab:planif}.										Analizando los datos de la misma obtenemos un error relativo medio de 14.02 \%.														
																												
\begin{table}
	\centering
		\begin{tabular}{|l|r|r|r|}
			\hline
			\textbf{Tarea} & \textbf{Est. Inicial} & \textbf{Est. Final} & \textbf{RE}\\ \hline \hline
			 Documentaci\'on sobre \LaTeX{}  & 25 d\'ias & 36 d\'ias & 30.5 \%  \\
			 Creaci\'on \texttt{pclass.cls} & 18 d\'ias & 19 d\'ias & 5.3 \% \\
			 Fase de pruebas y testeo  & 12 d\'ias & 14 d\'ias & 14.3  \% \\
			 Elaboraci\'on cap\'itulos memoria & 8 d\'ias & 10 d\'ias & 20 \% \\
			 Creaci\'on \texttt{pfcbibstyle}  & 5 d\'ias & 5 d\'ias & 0 \% \\		
			\hline
		\end{tabular}
	\caption{Planificaci\'on del proyecto}
	\label{tab:planif}
\end{table}



\section{Costes de desarrollo}
	En esta secci\'on estudiaremos detalladamente los costes que han supuesto la creaci\'on de la clase \texttt{pclass.cls}.
	Comenzaremos diciendo que no ha sido necesaria inversi\'on alguna destinada a la adquisici\'on de dispositivos 
	espec\'ificos, hardware o software. Este hecho ha supuesto una gran ventaja a la hora de elaborar el proyecto que nos 
	ocupa, ya que los costes se centrar\'an unicamente en los campos de costes de personal y costes indirectos.
	
	
	\subsection{Inversiones}
	En lo que respecta a la inversi\'on para la adquisici\'on de dispositivos espec\'ificos, tenemos que decir que en 
	nuestro caso dicha inversi\'on ha sido nula. Esto se debe a que no requeriamos de ning\'un material espec\'ifico para
	materializar la clase \texttt{pclass}.
	
		
	\subsection{Costes de software}
	En lo que respecta los costes de software usado para el desarrollo del proyecto, hemos de indicar primeramente
	que todo el software empleado es free software, o lo que es lo mismo software libre. Esto  supondr\'a un coste cero
	en lo que respecta a este apartado.
	
	Adem\'as del coste cero, el sofware libre tambi\'en nos proporciona muchas otras ventajas. Este tipo de software 
	brinda libertad a los usuarios sobre su producto adquirido y por tanto, una vez obtenido, puede ser usado, copiado,
	estudiado, modificado y redistribuido libremente. Seg\'un la Free Software Foundation, el software libre se refiere 
	a la libertad de los usuarios para ejecutar, copiar, distribuir, estudiar, cambiar y mejorar el software; de modo 
	m\'as preciso, se refiere a cuatro libertades de los usuarios del software: la libertad de usar el programa, 
	con cualquier prop\'osito; de estudiar el funcionamiento del programa, y adaptarlo a las necesidades; de distribuir 
	copias, con lo que puede ayudar a otros; de mejorar el programa y hacer p\'ublicas las mejoras, de modo que toda la 
	comunidad se beneficie. El software libre suele estar disponible gratuitamente, o al precio de coste de la distribuci\'on 
	a trav\'es de otros medios; sin embargo no es obligatorio que sea as\'i, por ende no hay que asociar la idea de 
	software libre a software gratuito.  
	 
 	Entre el software usado podemos citar entre otros:
 	
\begin{itemize}
	\item \TeX{}Live, en caso de que usemos Linux.
	\item Mik\TeX{}, en caso de usar Windows.
	\item Kyle, como editor \LaTeX{} sobre Linux.
	\item \TeX{}nicCenter, como editor \LaTeX{} sobre Windows.
	\item GhostScript y GSview, como visores de documentos.
\end{itemize}
 	
	En lo que respecta a la variedad de paquetes utilizados para complementar a la clase \texttt{pclass}, todos ellos
	son propuestas de software libre bajo licencia GNU, implicando tambien un coste nulo.
	
	
	\subsection{Costes de hardware}
	En el proceso de desarollo ha sido necesaria la adquisici\'on de dos equipos inform\'aticos. Concretamente hemos 
	hecho uso de dos PC port\'atiles Dell XPS 1530, los cuales hemos incluido como costes de hardware en el presupuesto.
	El valor que representan estos equipos es de 999 euros/unidad.
	
	
	\subsection{Costes indirectos}
	Dentro de esta secci\'on dedicada a los costes indirectos incluiremos el gasto producido por material consumible 
	(CD, impresora, papel, etc), luz consumida, limpieza de la zona de desarrollo, etc. Hemos considerado adecuado 
	 que un incremento del 5 \% en el presupuesto total ser\'ia representativo de este gasto.

	
	\subsection{Costes de personal}
	
	Para calcular el coste personal hemos utilizado como referencia los sueldos brutos estipulados m\'inimos 
	fijados en el BOE del 21 de Marzo de 2007, en el Anexo III. El sueldo anual para un Titulado de grado medio 
	es de 14637.56 euros \'o 6.53 $\frac{euros}{hora}$. 
	
	Tambi\'en debemos tener en cuenta el n\'umero total de horas dedicadas a la elaboraci\'on de este proyecto. Dicho 
	dato tiene un valor de 672 horas, representado por el trabajo realizado por dos personas durante 84 d\'ias dedicando 
	una media de 4 horas/d\'ia.
	
	De esta forma el coste personal supondr\'a una cuant\'ia de:
	
\begin{center}
$672 horas \cdot 6.53 \frac{euros}{hora} = 4388.16 euros$.
\end{center}
	
	
	\subsection{Presupuesto}
	
	Haciendo uso de los costes especificados anteriormente, obtendremos el presupuesto mostrado en la 
	Tabla~\ref{tab:presup}.

\begin{table}
	\centering
		\begin{tabular}{|p{7cm}|r|}
		\hline
		Inversiones & 0 euros \\
		Costes Software & 0 euros \\
		Costes Hardware & 1998 euros \\
		Costes Personal & 4388.16 euros \\
		Costes Indirectos & 319.31 euros \\ \hline  
		\textbf{TOTAL} & 6705.47 euros \\	
		\hline	
		\end{tabular}
	\caption{Presupuesto del proyecto}
	\label{tab:presup}
\end{table}

		
	
	