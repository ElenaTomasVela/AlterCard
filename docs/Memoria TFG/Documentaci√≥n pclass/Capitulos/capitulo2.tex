\chapter{Conceptos b\'asicos}	\label{conceptos} 
                                                 
\section{Primeros pasos en \LaTeX{}} 
	Desde el punto de vista del usuario, \LaTeX{} se presenta como un programa de l\'inea de comandos que toma 
	como par\'ametro 	principal el fichero fuente que contiene la descripci\'on (texto y comandos \LaTeX{}) del 
	documento a generar. Existen dos comandos para ejecutar \LaTeX{}:
	
	\begin{enumerate}
		\item \texttt{latex} genera el documento final en formato DVI (DeVice Independent), a partir del cual puede obtenerse, 
		mediante la aplicaci\'on \texttt{dvips}, el documento en formato PS (PostScript):
		
		\texttt{latex fichero.tex} (genera fichero.dvi, necesita un visualizador espec\'ifico, pero por la naturaleza 
		del formato, la lectura resulta lenta)
		
		\texttt{dvips fichero.dvi -o fichero.ps} (genera fichero.ps, m\'as manejable que DVI y directamente entendible por 
		muchas impresoras l\'aser, se puede ver con GhostScript/GhostView)
		
		\item \texttt{pdflatex} genera el documento directamente en formato PDF (Portable Document File, de uso muy extendido 
		en Internet):
		
		\texttt{pdflatex fichero.tex} (genera fichero.pdf, visualizable con Adobe Acrobat Reader) 
	\end{enumerate}
	
	A la hora de incluir gr\'aficos o im\'agenes en los documentos, hay que tener en cuenta que cada una de estas aplicaciones 
	es capaz de comprender s\'olo unos ciertos formatos gr\'aficos, si necesit\'asemos insertar figuras almacenadas en otros 
	formatos no directamente soportados, tendremos que recurrir a un programa que haga la conversi\'on. Por una 
	parte \texttt{latex} trabaja c\'omodamente tan s\'olo con EPS (Encapsulated PostScript, una variante especial de PS), 
	sin embargo \texttt{pdflatex} espera que las figuras est\'en en PDF (preferible para los gr\'aficos vectoriales), 
	PNG (adecuado para las capturas de pantalla o cualquier imagen generada por ordenador) o JPG (adecuada para fotograf\'ias).

	Una vez conocemos todo lo mencionado en el p\'arrafo anterior, ?`c\'omo hago uso de esos comandos en mi sistema operativo? 					\LaTeX{} puede usarse en Linux (y otros sistemas tipo UNIX) y en MS Windows (aunque parezca sorprendente). 
	Dependiendo del sistema operativo, la distribuci\'on y el m\'etodo de instalaci\'on var\'ia.
	
		\subsection{Instalaci\'on de \LaTeX{} y \TeX{}}
	
				\subsubsection{\LaTeX{} en Windows} %%%sacado de udla tesis + recetario latex
				La versi\'on m\'as popular de \LaTeX{} para Windows se llama MiK\TeX{} y la puedes bajar desde 
				\url{http://www.miktex.org}. Desde ah\'i bajas un Setup Wizard que, una vez instalado, se conecta a internet 
				para bajar e instalar el resto del programa. Realmente se instala una versi\'on reducida (guiada por asistente 
				al t\'ipico estilo Windows), pero luego se pueden descargar aquellos m\'odulos LaTeX (denominados paquetes) que 
				se vayan necesitando mediante MiK\TeX{} Options, que viene incluido en la distribuci\'on. La distribuci\'on 
				de MiK\TeX{} se puede ir actualizando mediante MiK\TeX{} Update Wizard.
	
				Es muy recomendable, si quieres generar y ver tu tesis en el formado PDF, que tengas instalado 
				Adobe Acrobat Reader en tu ordenador. Lo m\'as normal es que ya lo tengas instalado pero, si no lo tienes, 
				puedes bajarlo desde \url{http://www.adobe.com/products/acrobat}.
	
				Necesitas tambi\'en los programas AFPL Ghostscript y GSview para poder manipular archivos PostScript. 
				Ambos programas los puedes conseguir en la p\'agina de Internet \url{http://www.cs.wisc.edu/~ghost/}.

				Por \'ultimo, tambi\'en es muy recomendable que bajes el \TeX{}nicCenter. Es un editor de texto especializado 
				para \LaTeX{} con botones y ventanas, muy intuitivo y f\'acil de usar. Este programa, altamente recomendable, 
				lo puedes bajar en la direcci\'on \url{http://www.toolscenter.org/products/texniccenter/}.

				Es importante que el \'ultimo programa que instales sea \TeX{}nicCenter. Ya que, al iniciarlo la primera vez, 
				buscar\'a donde tienes instalados MiK\TeX{} y el resto de las aplicaciones para configurar todas las opciones 
				necesarias de manera autom\'atica. Una vez instalado \TeX{}nicCenter, s\'olo hay que realizar unos sencillos pasos 
				de configuraci\'on, consistentes en confirmar la ruta de acceso a MiK\TeX{} e indicar que se va a usar
				PDF (preferible, aunque tambi\'en puede ser DVI o PS) como formato de salida, esto har\'a que \TeX{}nicCenter 
				ejecute autom\'aticamente el comando \LaTeX{} apropiado.
			
				\subsubsection{\LaTeX{} en Linux} 
				Hay que instalar el paquete TeTex. \'Este incluye todo lo necesario, excepto el editor para poder escribir 
				los documentos LaTeX. Sin embargo el paquete TeTex no ha sido mantenido en mucho tiempo. Esto ha llevado a buscar 
				una soluci\'on, \'esta se llama TexLive, que incorpora soluciones a bugs y mejoras respecto a su antecesor.
				Como editor puede usarse:
			
	\begin{itemize}
	\item \textbf{Emacs} (paquete emacs), que dispone de un modo de edici\'on especial para \LaTeX{}, realzando 
	los comandos. Puede ser conveniente evaluar una extensi\'on para emacs denominada AUCTeX, que indenta 
	autom\'aticamente, con lo cual se obtiene una mejora ostensible de la legibilidad del c\'odigo, entre 
	otras cosas.
	\figura{0.9}{img/Emacs}{Emacs}{fig:emacs}{}			
	\item \textbf{Kile}, en el cual dispodemos de autocompletado de comandos \LaTeX{}, coloreado de sintaxis, 
	Kile autom\'aticamente marca los comandos \LaTeX{} y resalta los parentesis, y puede trabajar 
	con m\'ultiples ficheros a la vez. Adem\'as tambi\'en proporciona plantillas y patrones para facilitar 
	la creaci\'on de documentos.
  	\figura{0.9}{img/kile}{Kile}{fig:kile}{}
	\end{itemize}
			
			
\section{Lo que necesita saber sobre \LaTeX{}}%%%%%sacado de libros latex2.zip ---> una descripcion de latex2e
		
		\subsection{Ficheros de entrada de \LaTeX{}}
		El fichero de entrada para \LaTeX{} es un fichero de texto en formato ASCII. Se puede crear con cualquier editor 
		de textos. Contiene tanto el texto que se debe imprimir como las instrucciones, con las cuales \LaTeX{} interpreta 
		c\'omo debe disponer el texto. 
				
				\subsubsection{Signos de espacio}
				Los caracteres invisibles, como el espacio en blanco, el tabulador y el final de l\'inea, son tratados por 
				\LaTeX{} como signos de espacio propiamente dichos. Varios espacios seguidos se tratan como un espacio en 
				blanco. Generalmente, un espacio en blanco al comienzo de una l\'inea se ignora, y varios renglones en blanco 
				se tratan como un rengl\'on en blanco. 

				Un rengl\'on en blanco entre dos l\'ineas de texto definen el final de un p\'arrafo. Varias l\'ineas en blanco 
				se tratan como una sola l\'inea en blanco. El texto que mostramos a continuaci\'on es un ejemplo. A la derecha 
				se encuentra el texto del fichero de entrada y a la izquierda la salida formateada.\\
				
%%%%%%%%%%%%%% FIGURA PAG17 No importa si introduce varios espacios tras una palabra...%%%%%%%%%%%%%%%%%  
\begin{minipage}{0,4\linewidth}
No importa si introduce varios espacios
tras una palabra.

Con una l\'inea vac\'ia se empieza un
nuevo p\'arrafo.
\end{minipage}
\hfill\begin{minipage}{0,5\linewidth}
\begin{verbatim}
No importa si introduce
varios      espacios tras
una palabra.

Con una línea vacía se empieza 
un nuevo párrafo.
\end{verbatim}
\end{minipage}
				
				\subsubsection{Caracteres especiales}
				Los s\'imbolos siguientes son caracteres reservados que tienen un significado especial para \LaTeX{} o 
				que no est\'an disponibles en todos los tipos. Si los introduce en su fichero directamente es muy probable 
				que no se impriman o que fuercen a \LaTeX{} a hacer cosas que Vd. no desea.\\
								\begin{center}\verb+$ & % # _ { } ~ ^ \+ \end{center}
				Como puede ver, algunos de estos caracteres se pueden incluir en sus documentos anteponiendo el car\'acter 
				(backslash) \verb+\+:
				 				\begin{center}\verb+\$ \& \% \# \_ \{ \}+ \end{center}
				 				
				Los restantes s\'imbolos y otros muchos caracteres especiales se pueden imprimir en f\'ormulas matem\'aticas 
				o como acentos con \'ordenes espec\'ificas.
				
				\subsubsection{Las \'ordenes de \LaTeX{}}
				En las \'ordenes \LaTeX{} se distinguen las letras may\'usculas y las min\'usculas. Toman uno de los dos 
				formatos siguientes:
						\begin{itemize}
								\item Comienzan con un backslash y tienen un nombre compuesto s\'olo por letras. Los nombres de las 
								\'ordenes acaban con uno o m\'as espacios en blanco, un car\'acter especial o una cifra.
								
								\item Se compone de un backslash y un car\'acter especial. 
						\end{itemize}
 
				\LaTeX{} ignora los espacios en blanco que van tras las \'ordenes. Si se desea introducir un espacio en blanco 
				tras una instrucci\'on, se debe poner o bien \{\} y un espacio, o bien una instrucci\'on de espaciado despu\'es 
				de la orden. Con \{\} se fuerza a \LaTeX{} a dejar de ignorar el resto de espacios que se encuentren despu\'es 
				de la instrucci\'on.\\
				
%%%%%%%%%%%%%%FIGURA PAG 17 He leido que Knuth distingue a la gente ...%%%%%%%%%%%%%%%%%%%%%%%%%%%%%%%%%
\begin{minipage}{0,4\linewidth}
He le\'ido que Knuth distingue 
a la gente que trabaja con 
\TeX{} en \TeX{}nicos y \TeX{}pertos.\\
Hoy es \today.
\end{minipage}
\hfill\begin{minipage}{0,5\linewidth}
\begin{verbatim}
He le'ido que Knuth distingue 
a la gente que trabaja con 
\TeX{}  en \TeX{}nicos 
y \TeX{}pertos.\\
Hoy es \today.
\end{verbatim}
\end{minipage}\\
\\

				Algunas instrucciones necesitan un par\'ametro que se debe poner entre llaves \{\} tras la instrucci\'on. Otras 
				\'ordenes pueden llevar par\'ametros opcionales que se a\~naden a la instrucci\'on entre corchetes [ ] o no. 


				\subsubsection{Comentarios}
				Cuando \LaTeX{} encuentra un car\'acter \% mientras procesa un fichero de entrada, ignora el resto de la 
				l\'inea. Esto suele ser \'util para introducir notas en el fichero de entrada que no se mostrar\'an en la 
				versi\'on impresa.\\
%%%%%%%%%%%%%%%%%%%%%%%%% FIGURA PAG 18 Esto es un ejemplo tonto...%%%%%%%%%%%%%%%%%%%%%%%%%%%%%%%%%%%%%%%%%%

\begin{minipage}{0,4\linewidth}
Esto es un % tonto 
% o mejor instructivo
ejemplo.
\end{minipage}
\hfill\begin{minipage}{0,5\linewidth}
\begin{verbatim}
Esto es un % tonto 
% o mejor instructivo
ejemplo.
\end{verbatim}
\end{minipage}\\
\\

				Esto a veces puede resultar \'util cuando nos encontramos con l\'ineas demasiado largas en el fichero fuente. 
				Si no quisi\'esemos introducir un espacio entre dos palabras, y preferimos tener dos renglones, entonces el signo 
				\% debe ir justo al final del rengl\'on pero pegado al \'ultimo car\'acter. De este modo comentamos el car\'acter 
				de salto de l\'inea, que se hubiese tratado como un espacio en blanco.\\ 
%%%%%%%%%%%%%%%%%%%%%%%%% FIGURA PAG 18 Esto es otro ejemplo...%%%%%%%%%%%%%%%%%%%%%%%%%%%%%%%%%%%%%%%%%%%%%

\begin{minipage}{0,4\linewidth}
Este es otro ejem% y
% ahora el resto
plo.
\end{minipage}
\hfill\begin{minipage}{0,5\linewidth}
\begin{verbatim}
Este es otro ejem% y
% ahora el resto
plo.
\end{verbatim}
\end{minipage}\\
\\

		\subsection{Estructura de un fichero de entrada}
		Cuando \LaTeXe{} procesa un fichero de entrada, espera de el que siga una determinada estructura. Todo 
		fichero de entrada debe comenzar con la orden\\ 
				\verb+\documentclass{...}+\\
		
		Esto indica qu\'e tipo de documento es el que se pretende crear. Tras esto, se pueden incluir \'ordenes que 
		influir\'an sobre el estilo del documento entero, o cargar paquetes que a\~nadir\'an nuevas propiedades al sistema 
		de \LaTeX{}. Para cargar uno de estos paquetes se usar\'a la instrucci\'on\\ 
				\verb+\usepackage{...}+\\
		Cuando todo el trabajo de configuraci\'on est\'e realizado entonces comienza el cuerpo del texto con la instrucci\'on\\
				\verb+\begin{document}+\\
		El \'area comprendida entre \verb+\documentclass{...}+ y \verb+\begin{document}+ recibe el nombre de pre\'ambulo, una 
		vez que \'este ha finalizado se introducir\'a el texto mezclado con algunas instrucciones \'utiles de \LaTeX{}. Por 
		\'ultimo para finalizar el documento debe ponerse la orden\\ 
				\verb+\end{document}+\\
		De esta forma \LaTeX{} ingorar\'a cualquier cosa que se ponga tras esta instrucci\'on. Las figuras \ref{fig:minimo} y
		\ref{fig:articulo} muestran el contenido m\'inimo de un fichero de \LaTeXe{}, as\'i como un fichero de entrada algo m\'as complicado.\\
	%%%%%%%%%%%%%%%%%%%%%%%%% FIGURA 1.1 y 1.2 PAG 19%%%%%%%%%%%%%%%%%%%%%%%%%%%%%%%%%%%%%%%%%%%%%%%%%%%%%%%%%
	
\begin{figure}[htbp]
\centering
\begin{tabular}{lp{10cm}}
	\hline
		\verb+\documentclass{article}+\\
		\verb+\begin{document}+\\
		\verb+Lo bueno si es breve, dos veces bueno.+\\
		\verb+\end{document}+\\	
		\verb+\end{verbatim}+\\
	\hline
\end{tabular}
\caption{Fichero m\'inimo de \LaTeX{}}
\label{fig:minimo}
\end{figure}

\begin{figure}[htbp]
\centering
\begin{tabular}{lp{10cm}}
	\hline
		\verb+\documentclass[a4paper,11pt]{article}+\\
		\verb+\usepackage{latexsym}+\\
		\verb+\usepackage[activeacute,spanish]{babel}+\\
		\verb+\author{H.~Partl}+\\	
		\verb+\title{Minimizando} +\\
		\verb+\frenchspacing+\\
		\verb+\begin{document}+\\
		\verb+\maketitle+\\
		\verb+\tableofcontents+\\
		\verb+\section{Inicio}+\\
		\verb+Aqu'i comienza un art'iculo estupendo.+\\
		\verb+\section{Fin}+\\
		\verb+Aqu'i acaba mi art'iculo.+\\
		\verb+\end{document}+\\
	\hline
\end{tabular}
\caption{Ejemplo de art\'iculo cient\'ifico en espa\~nol}
\label{fig:articulo}
\end{figure}
		
		\subsection{El formato del documento}
				\subsubsection{Clases de documetos}
				Cuando procesa un fichero de entrada, lo primero que necesita saber \LaTeX{} es el tipo de documento que el 
				autor quiere crear. Esto se indica con la instrucci\'on\\
							\verb+\documentclass[opciones]{clase}+\\
				En este caso, la clase indica el tipo de documento que se crear\'a. En el cuadro \ref{tab:clases} se muestran 
				las clases de documento que explicaremos m\'as adelante. La distribuci\'on de \LaTeXe{} proporciona m\'as clases para 
				otros documentos, como cartas y transparencias. El par\'ametro de opciones personaliza el comportamiento de la 
				clase de documento elegida. Las opciones se deben separar con comas. En el cuadro \ref{tab:opciones} se indican 
				las opciones m\'as comunes de las clases de documento est\'andares. 
				
				Por ejemplo: un fichero de entrada para un documento de \LaTeX{} podr\'ia comenzar con\\ 
							\verb+\documentclass[11pt,twoside,a4paper]{article}+\\
				Esto le indica a \LaTeX{} que componga el documento como un art\'iculo utilizando tipos del cuerpo 11, y que 
				produzca un formato para impresi\'on a doble cara en papel DIN A4.\\
%%%%%%%%%%%%%%%%%%%%%%%%%%%%%%%%TABLA 1.1 PAG 20 Y 1.2 PAG 22%%%%%%%%%%%%%%%%%%%%%%%%%%%%%%%%%%%%%%%%%%%%%%%%%%%%

\begin{table}[htbp]
\centering
	\hrulefill
	\begin{description}
	\item[article] para art\'iculos de revistas especializadas, ponencias, trabajos de pr\'acticas de formaci\'on, 
	trabajos de seminarios, informes peque\~nos, solicitudes, dict\'amenes, descripciones de programas, 
	invitaciones y muchos otros.
	\item[report] para informes mayores que constan de m\'as de un cap\'itulo, proyectos fin de carrera, tesis doctorales, 
	libros peque\~nos, disertaciones, guiones y similares.
	\item[book] para libros de verdad.
	\item[slide] para transparencias. Esta clase emplea tipos grandes sans serif.
	\end{description}
	\hrulefill
\caption{Clases de documentos}
\label{tab:clases}
\end{table}

\begin{table}[htbp]
\centering
	\hrulefill
	\begin{description}
	\item[10pt, 11pt, 12pt] Establecen el tama\~no (cuerpo) para los tipos. Si no se especifica ninguna opci\'on, 
	se toma \textbf{10pt}.
	\item[a4paper, letterpaper, \ldots] Define el tama\~no del papel. Si no se indica nada, se toma \textbf{letterpaper}. 
				Aparte de \'este se puede elegir \textbf{a5paper}, \textbf{b5paper}, \textbf{executivepaper} y \textbf{legalpaper}.
	\item[fleqn] Dispone las ecuaciones hacia la izquierda en vez de centradas.
	\item[leqno] Coloca el n\'umero de las ecuaciones a la izquierda en vez de a la derecha.
	\item[titlepage, notitlepage] Indica si se debe comenzar una p\'agina nueva tras el t\'itulo del documento o no. Si no 
				se indica otra cosa, la clase \textbf{article} no comienza una p\'agina nueva, mientras que \textbf{report} y 
				\textbf{book} s\'i.
	\item[twocolumn] Le dice a \LaTeX{} que componga el documento en dos columnas.
	\item[twoside, oneside] Especifica si se debe generar el documento a una o a dos caras. En caso de no indicarse otra 
				cosa, las clases \textbf{article} y \textbf{report} son a una cara y la clase \textbf{book} es a dos.
	\item[openright, openany] Hace que los cap\'itulos comienzen o bien s\'olo en p\'aginas a la derecha, o bien en la 
				pr\'oxima que est\'e disponible. Esto no funciona con la clase \textbf{article}, ya que en esta clase no 
				existen cap\'itulos. De modo predeterminado, la clase \textbf{report} comienza los cap\'itulos en la pr\'oxima 
				p\'agina disponible y la clase \textbf{book} las comienza en las p\'aginas a la derecha.
	\end{description}
	\hrulefill
\caption{Opciones de clases de documentos}
\label{tab:opciones}
\end{table}
  
				
				\subsubsection{Paquetes}
				En algunas situaciones el \LaTeX{} b\'asico no es suficiente, por ejemplo si queremos incluir gr\'aficos, texto 
				en color o el c\'odigo fuente de un fichero, necesita mejorar las capacidades de \LaTeX{}. Tales mejoras son 
				conocidas como paquetes. Los paquetes se activan con la orden\\
							\verb+\usepackage[opciones]{paquete}+\\
				donde paquete es el nombre del paquete que queremos usar y opciones es una lista palabras clave que activan 
				funciones especiales del paquete. Algunos paquetes vienen con la distribuci\'on b\'asica de \LaTeX{} (v\'ease la 
				Cuadro~\ref{tab:paquetes}). Otros se porporcionan por separado. En la Gu\'ia Local \cite{guiaa} puede encontrar m\'as 
				informaci\'on sobre los paquetes disponibles en su instalaci\'on local. La fuente principal de informaci\'on 
				sobre \LaTeX{} es The \LaTeX{} Companion \cite{latexcomp}. Contiene descripciones de cientos de paquetes, as\'i como 
				informaci\'on sobre c\'omo escribir sus propias extensiones a \LaTeXe{}.\\
%%%%%%%%%%%%%%%%%%%%%%%%%%%%%INSERTAR TABLA 1.3 PAG 22%%%%%%%%%%%%%%%%%%%%%%%%%%%%%%%%%%%%%%%%%%%%%%%%%%%%%%%%%%%%%%%%%%%%

\begin{table}[htbp]
\centering
	\hrulefill
	\begin{description}
	\item[doc] Permite la documentaci\'on de paquetes y otros ficheros de \LaTeX{}.\\ Se describe en \texttt{doc.dtx}. 
	\item[exscale] Proporciona versiones escaladas de los tipos adicionales para matem\'aticas.\\
					Descrito en \texttt{ltexscale.dtx}.  
	\item[fontenc] Especifica qu\'e codificaci\'on de tipo debe usar \LaTeX{}.\\
					Descrito en \texttt{ltoutenc.dtx}.
	\item[ifthen] Proporciona instrucciones de la forma `si. . . entonces. . . si no. . . '\\
					Descrito en \texttt{ifthen.dtx}.
	\item[latexsym] Para que \LaTeX{} acceda al tipo de s\'imbolos, se debe usar el paquete \texttt{latexsym}.\\
					Descrito en \texttt{latexsym.dtx}.
	\item[makeidx] Proporciona instrucciones para producir \'indices de materias.
	\item[syntonly] Procesa un documento sin componerlo, lo cual es \'util para la verificaci\'on r\'apida de errores.\\
					Se describe en \texttt{syntonly.dtx}.
	\item[inputenc] Permite la especificaci\'on de una codificaci\'on de entrada como ASCII (con la opci\'on ascii), 
					ISO Latin-1 (con la opci\'on latin1), ISO Latin-2 (con la opci\'on latin2), p\'aginas de c\'odigo de 
					437/850 IBM (con las opciones cp437 y cp580, respectivamente), Apple Macintosh (con la opci\'on applemac), 
					Next (con la opci\'on next), ANSI-Windows (con la opci\'on ansinew) o una definida por el usuario.\\ 
					Descrito en \texttt{inputenc.dtx}.
	\end{description}
	\hrulefill
\caption{Algunos paquetes distribuidos con \LaTeX{}}
\label{tab:paquetes}
\end{table}
		
				\subsubsection{Estilo de p\'agina}
				Con \LaTeX{} existen tres combinaciones predefinidas de cabeceras y pies de p\'agina, a las que se llaman 
				estilos de p\'agina. El par\'ametro estilo de la instrucci\'on\\
							\verb+\pagestyle{estilo}+\\
				define cu\'al usarse. La Cuadro~\ref{tab:estilos} muestra los estilos de p\'agina predefinidos.\\ 
%%%%%%%%%%%%%%%%%%%%%%%%%%%%%TABLA 1.4 PAG 23%%%%%%%%%%%%%%%%%%%%%%%%%%%%%%%%%%%%%%%%%%%%%%%%%%%%%%%%%%%%%%%%%%%%

\begin{table}[htbp]
\centering
	\hrulefill
	\begin{description}
	\item[plain] imprime los n\'umeros de p\'agina en el centro del pie de las p\'aginas. Este es el estilo de 
				p\'agina que se toma si no se indica ning\'un otro. 
	\item[headings] en la cabecera de cada p\'agina imprime el cap\'itulo que se est\'a procesando y el n\'umero de 
				p\'agina, mientras que el pie est\'a vac\'io. (Este estilo es similar al empleado en este documento). 
	\item[empty] deja tanto la cabecera como el pie de las p\'aginas vac\'ios.
	\end{description}
	\hrulefill
\caption{Estilos de p\'agina predefinidos en \LaTeX{}}
\label{tab:estilos}
\end{table}

				Es posible cambiar el estilo de p\'agina de la p\'agina actual con la instrucci\'on\\
							\verb+\thispagestyle{estilo}+\\
				En The \LaTeX{} Companion [3]               hay una descripci\'on de c\'omo crear sus propias cabeceras y 
				pies de p\'agina.
						 
		\subsection{Proyectos grandes}
		Cuando trabaje con documentos grandes, podr\'ia, si lo desea, dividir el fichero de entrada en varias partes. 
		\LaTeX{} tiene varias instrucciones que le ayudan a realizar esto.\\
				\verb+\include{fichero}+\\
		Se puede utilizar en el cuerpo del documento para introducir el contenido de otro fichero. Observe que 
		\LaTeX{} comenzar\'a una p\'agina nueva antes de procesar el texto del fichero.
		
		La siguiente instrucci\'on solo puede ser utilizada en el pre\'ambulo. Permite indicarle a \LaTeX{} que s\'olo tome 
		la entrada de algunos ficheros de los indicados con \verb+\include+.\\
				\verb+\includeonly{fichero, fichero,...}+\\
		Una vez que esta instrucci\'on se ejecute en el pre\'ambulo del documento, s\'olo se procesar\'an las 
		instrucciones \verb+\include+ con los ficheros indicados en el argumento de la orden \verb+\includeonly+. Observe 
		que no hay espacios entre los nombres de fichero y las comas. 
		
		Existe otra opci\'on que puede ser usada a lo largo del documento para introducir el texto situado en el 
		fichero correspondiente. En este caso no se produce ning\'un salto de p\'agina ni de l\'inea a menos que se indique 
		adecuadamente.\\
				\verb+\input{fichero}+\\
		Resulta muy \'util para dividir un trabajo en diferentes archivos y agruparlos en un documento donde tendremos 
		un pre\'ambulo com\'un a todos. 
					 	

\section{?`Qu\'e documentaci\'on hay sobre \TeX{}/\LaTeX{}?}
	\subsection{Tutoriales}%%%%sacado de libros/esLatex.zip
 	\subsubsection{Tutoriales en castellano}
								
	\begin{itemize}
	\item \textbf{Una descripci\'on de \LaTeX{}} Tom\'as Bautista. Este documento se encuentra en CTAN (en 
 	CTAN:documentation/short/spanish)
					
	\item \textbf{Bases de datos bibliogr\'aficos, \LaTeX{} y el idioma espa\~nol} Luis Seidel:
         \url{ftp://tex.unirioja.es/pub/tex/doc/bibliogr.pdf}
					
	\item \textbf{Recetario para \LaTeX{}} por Aristarco. Disponible en:\\
	\url{http://recetariolatex.cjb.net}

	\end{itemize}
 				
 	\subsubsection{Tutoriales en otros idiomas}%%%%sacado de libros/esLatex.zip
 										
	\begin{itemize}
	\item \textbf{A Gentle Introduction to \TeX{}} de Michael Dobb, 
	disponible en :\url{CTAN:documentation/gentle}

	\item \textbf{Simplied Introduction to \LaTeX{}} de Harvey J. Greenberg,
	disponible en :\url{CTAN:documentation/simplified-latex/latex.ps}
															
	\item \textbf{The not so Short Introduction to \LaTeX{}} de Tobias Oetiker,
	disponible en :\url{CTAN:documentation/lshort/} 
					
								\end{itemize}
 		
 								
 		\subsection{Libros}%%%%sacado de libros esLatex.zip
 						\subsubsection{Libros en castellano}
 															
								\begin{itemize}
											\item \textbf{El libro de \LaTeX{}} \cite{librolatex}. Bernardo Cascales, Pascual Lucas, Jos\'e 
											Manuel Mira, Antonio Pallar\'es y Salvador S\'anchez-Pedre\~no. Prentice Hall, Madrid, 2003.
											Se divide en dos partes: en la primera, que consta de 18 lecciones,  el lector encontrar\'a todo 
											lo que necesita para componer documentos b\'asicos con \LaTeX{}, que incluso pueden llegar a 
											alcanzar una complejidad tipogr\'afica notable. En la segunda parte, que tiene intenci\'on 
											de servir como manual de referencia y uso avanzados, profundiza en todos los aspectos tratados 
											en las lecciones y a\~nade otros nuevos.
											
											\item \textbf{Iniciaci\'on a \LaTeXe{}} \cite{botella}. Javier Sanguino Botella, Addison-Wesley (1997).
											Uno de los mejores manuales para empezar a aprender.
											
											\item \textbf{Composici\'on de textos cient\'ificos con \LaTeX{}} \cite{valiente}. G. Valiente. 
											Edicions UPC, Barcelona, 1997.
											Pretende dar a conocer el sistema \LaTeX{} en el contexto de la composici\'on de textos 
											cient\'ificos. As\'i, puede servir de introducci\'on a aquellos estudiantes que se inicien en 
											la escritura cient\'ifica y, a la vez, puede resultar un texto de consulta permanente para 
											profesores e investigadores. Instructivo y f\'acil de leer, puede adquirirse en versi\'on 
											papel o en formato electr\'onico, a trav\'es del web de la editorial Edicions 
											UPC \url{www.edicionsupc.es}.
											
											\item \textbf{\LaTeX{}, una imprenta en sus manos} \cite{imprenta}. Bernardo Cascales Salinas.
											Aula Documental de Investigaci\'on, 2000.
											Est\'a escrito a modo enciclop\'edico y toda la informaci\'on sobre un tema est\'a tratada en 
											un \'unico lugar. Extensa obra que describe tanto tanto los comandos b\'asicos como las 
											funcionalidades m\'as avanzadas de \LaTeX{}. Se incluye una descripci\'on de la programaci\'on 
											de macros y generaci\'on de documentos para su publicaci\'on en Internet. Se muestran
											interesantes ejemplos de uso en campos alejados de las matem\'aticas, como puede ser la
											composici\'on tipogr\'afica en otros alfabetos, s\'imbolos qu\'imicos e incluso partituras 
											musicales.
 
								\end{itemize}
								
						\subsubsection{Libros en otros idiomas}
									
								\begin{itemize}
											\item \textbf{A Guide to \LaTeX{}} \cite{aguide}. H. Kopka y P.W. Daly, Addison-Wesley 
											Professional (2004).
											Probablemente, el mejor manual existente sobre \LaTeX{}. Contiene una gu\'ia completa de 
											\'ordenes, abundantes ejemplos e informaci\'on adicional. (Incluye las dos versiones en uso 
											de \LaTeX{}, \LaTeXe{} y la m\'as antigua, \LaTeX{}2.09).
											
											\item \textbf{The \LaTeX{} Companion.} \cite{latexcomp}. Frank Mittelbach, Michel Goossens, 
											Johannes Braams, David Carlisle, Chris Rowley, Addison Wesley Professional (2004).
											Este manual sirve de ampliaci\'on a \cite{aguide}. Es una recopilaci\'on de informaci\'on 
											sobre los llamados paquetes (packages), conjuntos de macros que distintos autores han puesto 
											a disposici\'on p\'ublica.
 
											\item \textbf{\LaTeX{}-A Document preparation system}\cite{adocument}. L. Lamport (dos ediciones) 
														Addison-Wesley (1985 y 1994).
														Escrito por Leslie Lamport (autor de \LaTeX{}) esto lo dice todo, contiene todo lo 
														necesario para iniciarse en el mundo \LaTeX{}. Puede resultar un tanto insuficiente para
														usuarios avanzados. 
														
								\end{itemize}
		
		\subsection{Varios}
					\begin{description}
											\item[Cervan\TeX{}] Grupo de usuarios hispanohablantes de \TeX{}. Esta asociaci\'on 
											busca intercambiar experiencias sobre \TeX{} y \LaTeX{}, as\'i como sobre sus aplicaciones, 
											y promoverlo de forma adecuada en el \'ambito hispanohablante (tanto Espa\~na como Am\'erica).
											\url{http://www.cervantex.es}
											
											\item[CTAN: The Comprehensive \TeX{} Archive Network] Esta p\'agina es un repositorio 
											de software y documentaci\'on relacionada con \TeX{}.\\ \url{http://www.ctan.org/}
											
											\item[The Prac\TeX{} Journal] Una revista de \TeX{} online y gratuita. 
											Disponible en\\  \url{http://dw.tug.org/pracjourn/} 
											
											\item[LaTeX-project]
											Aqui encontrar\'as material de referencia para aprender a utilizar y sacarle m\'as provecho a tu sistema 
											de \LaTeX{}.\\  \url{http://www.LaTeX-project.org} 
											
					\end{description}
											
					