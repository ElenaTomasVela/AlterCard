\chapter{Definici\'on de objetivos}\label{defobjetivos}

Los objetivos a cumplir durante la realización de este proyecto se pueden	
dividir en varias categorías.

\section{Objetivos principales}

\begin{itemize}
  \item Proporcionar un entorno de juego en el que los jugadores
    puedan divertirse.
  \item Permitir que los jugadores puedan interactuar y relacionarse
    durante las partidas
  \item Crear una experiencia customizable para los jugadores.
\end{itemize}

\section{Objetivos técnicos}
En el amplio mundo de la programación de páginas web, hay
muchas tecnologías entre las que escoger que no se han dado
durante el Grado. 
Por ello, se quiere usar esta oportunidad para
explorar las nuevas herramientas que se pueden usar para estos fines.

Especialmente, se quiere experimentar con el uso de:
\begin{description}
  \item[Bun:] Lanzado a finales de 2023, se trata de un conjunto
    de herramientas para el desarrollo en JavaScript, prometiendo
    un mayor rendimiento que Node.js, aunque manteniendo compatibilidad con
    sus paquetes. Entre sus características se encuentran la ejecución nativa
    de ficheros de TypeScript, ejecución de tests, y gestión de paquetes.
  
  \item[MongoDB:] Durante la carrera, se ha hecho mucho hincapié en bases de datos
    relacionales. Con la incorporación de MongoDB, se quiere explorar el mundo
    de las bases de datos no relacionales, además de hacer uso de sus puntos
    fuertes como el mejor rendimiento y escalabilidad, que hacen de ello una herramienta
    idónea para aplicaciones a tiempo real.
  
  \item[React:] Aunque este framework se haya usado en algunos de los proyectos
    del Grado, la popularidad de este framework es innegable, debido a que se sigue usando
    frecuentemente hoy en día. Adicionalmente, se quiere explorar su combinación con
    otras librerías de componentes y de estilos, para así ver las posibilidades
    que puede abarcar el diseño web.
\end{description}

\section{Objetivos formativos}
Adicionalmente, este proyecto pretende poner en práctica todos
los conocimientos adquiridos durante el Grado mediante la
abarcación de un proyecto mayor.

